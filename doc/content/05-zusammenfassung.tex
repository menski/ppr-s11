\section{Zusammenfassung}
\label{sec:zusammenfassung}

Das in dieser Arbeit vorgestellte Skript ermöglicht es, eine lokale Installation der Wikipedia mit Bildern anzureichern. Außerdem kann die lokale Installation anschließend auf weitere Server übertragen und dort eingerichtet werden. Damit ist die in Abschnitt \ref{sec:aufgabestellung} beschriebene Aufgabenstellung erfüllt. Das Skript ermöglicht über eine parallele Abarbeitung anhand von Prozessen, eine optimale Auslastung der zu Verfügung stehenden Rechenleistung. Außerdem ist das Skript durch die parallel Abarbeitung sehr ressourcensparend, was sich besonders bei großen Traces auswirkt. Trotzdem ist die Ausführung des Skripts mit allen Schritten immer noch sehr zeitintensiv. Dies resultiert aus den großen Datenmengen, welche verarbeitet, kopiert und verschoben werden müssen. 

Zur Stabilität des Skriptes kann keine klare Aussage getroffen werden. Aufgrund einer sehr kurzen Entwicklungszeit und mehrerer Hardwaredefekte konnte es kaum auf dem Produktiv-System getestet werden. Allerdings konnten mit Hilfe des Skriptes 4 Server, für ein 30 minütigen Trace-Ausschnitt, erfolgreich eingerichtet werden. Es liegen jedoch keine Messungen vor, wie hoch der aktuelle Anteil von nicht erreichbaren Ressourcen ist.

Für das Wiedereinspielen der Traces ist zu beachten, dass auch beim Anfordern von Wikipedia-Artikeln ein \texttt{404} Response erzeugt werden kann.
%Außerdem ist der Response auf ein nicht existierendes Thumbnail \texttt{500 Internal Server Error}. Das ist dadurch zu erklären, das wenn das Thumbnail nicht gefunden wird, das Mediawiki versucht dieses dynamisch aus dem originalen Bild zu erzeugen. Ist dieses Bild ebenfalls nicht vorhanden, kommt es zu einem internen Fehler auf dem Server.
Wird ein Thumbnail nicht gefunden und kann es nicht dynamsich aus dem originalen Bild erzeugt werden, da dieses z. B. nicht vorhanden ist, besitzt der Response den Statuscode \texttt{500 Internal Server Error}.

\section{Ausblick}
\label{sec:ausblick}

Wie bereits in der Zusammenfassung (siehe Abschnitt \ref{sec:zusammenfassung}) erwähnt, konnte das Skript kaum getestet werden. Hier bedarf es weiterer Praxis-Tests und gegebenenfalls Fehlerbeseitigung. Die Phasen während des Downloads und der Installation auf den Servern benötigten am meisten Zeit und Ressourcen. Diese Phasen könnten möglicherweise optimiert oder anders konzipiert werden. Außerdem ist es möglich, nach dem Download der Bilder den gefilterten Trace ein weiteres mal zu filtern und die nicht verfügbaren Bilder und Thumbnails zu entfernen.



%%% Local Variables: 
%%% mode: latex
%%% TeX-master: "../master"
%%% End: 

% LocalWords:  Traces Trace Wiedereinspielen Response Thumbnail Internal Error
% LocalWords:  Mediawiki Thumbnails
